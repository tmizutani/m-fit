\section{Processamento de Alto N�vel \label{alto_nivel}}

\subsection{M�todo Baseado em Entropia \label{entropia}}

\hspace*{1.25cm}Nascida na termodin�mica, a entropia tem como
objetivo, medir o grau de desordem de um sistema, em outras
palavras, determinar o quanto inst�vel se encontra um determinado
estado de um sistema.

\hspace*{0.65cm}Como visto em \cite{MENDELSSOHN_1962}, o conceito de
instabilidade, ou desordem de um sistema termodin�mico est�
associado � quantidade de troca de calor entre um estado A e outro
estado B. Isto porque a desordem de um sistema aumenta ou diminui
conforme for o ganho ou perda de calor na passagem de um estado para
outro.

\hspace*{0.65cm}Por exemplo, para passar a �gua do estado (tomada)
s�lido para o estado l�quido � necess�rio que o estado s�lido receba
uma determinada quantidade de calor $\Delta Q$. Quando inicia-se o
processo de aquecimento do cubo de gelo (transi��o), suas mol�culas,
inicialmente est�veis, passam a ter um aumento da energia cin�tica,
conseq�entemente tornando o sistema (v�deo) mais inst�vel. Esta
movimenta��o pode ser considerada como desordem, sendo assim, �
poss�vel dizer que no estado l�quido, a entropia do sistema � maior
do que quando o sistema se encontra no estado s�lido.

\hspace*{0.65cm}O m�todo baseado em entropia � um m�todo n�o muito
utilizado e neste trabalho este m�todo n�o ser� aplicado.

\hspace*{0.65cm}Uma abordagem sobre este assunto pode ser encontrado
em \cite{MENDELSSOHN_1962}.



%\begin{large}
%\begin{center}
%\begin{equation}
%\label{eq:entropia} Sj = - \sum_{i=0}^{n}{Pi * ln(Pi)}
%\end{equation}
%\end{center}
%\end{large}
