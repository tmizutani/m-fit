\chapter*{\centering GLOSS�RIO \label{glossario}}

\hspace*{1.25cm}\textbf{C}

\hspace*{0.65cm}CMYK - Sistema de cores onde suas letras representam
as cores Ciano (\textit{Cyan}), Magenta (\textit{Magenta}), Amarelo
(\textit{Yellow}) e Preto (\textit{blacK}).


\hspace*{0.65cm}\textbf{D}

\hspace*{0.65cm}\textit{Drag and Drop} - Do ingl�s, arrastar e
soltar.


\hspace*{0.65cm}\textbf{H}

\hspace*{0.65cm}HSV - Sistema de cores baseado nas seguintes
propriedades: Tonalidade (\textit{Hue}), Satura��o
(\textit{Saturation}), Valor(\textit{Value}).


\hspace*{0.65cm}\textbf{I}

\hspace*{0.65cm}ISO - A Organiza��o Internacional para Padroniza��o (em l�ngua inglesa: "International Organization for Standardization - ISO).
Popularmente conhecida como ISO, � uma entidade que atualmente congrega os gr�mios de padroniza��o/normaliza��o de 170 pa�ses.


\hspace*{0.65cm}\textbf{L}

\hspace*{0.65cm}Limiar - Par�metro utilizado como refer�ncia para
tomada de decis�es em opera��es que sejam dependentes da varia��o de
um determinado valor.


\hspace*{0.65cm}\textbf{P}

\hspace*{0.65cm}PDI - Processamento Digital de Imagens.

\hspace*{0.65cm}\textit{Preview} - Do ingl�s, significa pr�via,
amostra.


\hspace*{0.65cm}\textbf{Q}

\hspace*{0.65cm}QT - Ferramenta desenvolvida em linguagem C para
cria��o de interface de aplica��o.


\hspace*{0.65cm}\textbf{R}

\hspace*{0.65cm}RGB - Tipo de sistema de cores. Suas letras
representam Vermelho (\textit{RED}), Verde (\textit{GREEN})e Azul
(\textit{BLUE})


\hspace*{0.65cm}\textbf{T}

\hspace*{0.65cm}\textit{Timeline} - Do ingl�s, linha do tempo. Neste
trabalho a \textit{timeline} representa um resumo do v�deo. Onde
cada Frame mostrado na \textit{timeline} representa 1 segundo do
v�deo.
